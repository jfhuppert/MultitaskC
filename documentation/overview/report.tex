\documentclass[10pt]{report}
\usepackage{latex8}

\usepackage{amsmath,amssymb}
\usepackage{graphicx}%

\pagestyle{empty}

\newenvironment{todo}%
{\description \item[\fbox{To Do!}] \sl%
\typeout{----------- Something left to be done here! ------------------}%
}%
{\enddescription}%

\makeatletter
\def\captionof#1#2{{\def\@captype{#1}#2}}
\makeatother

\makeatletter
\long\def\unmarkedfootnote#1{{\long\def\@makefntext##1{##1}
\footnotetext{#1}}}
\makeatother

\sloppy

\begin{document}

\title{A compiled parallelism programming language : MultitaskC}

\author{
Hugo Delchini\\
Email: hugo@delchini.fr\\
}

\maketitle
\thispagestyle{empty}

\begin{abstract}

Computing applications are often written using a high level 
computer language for programming and an operating system for 
execution handling. A family of programming languages (CPL for 
"Compiled Parallelism Languages") offers features and advantages 
similar to a programming language coupled with a multi-task 
operating system. We intend here to point out that CPLs bring 
several additional advantages in terms of performance, especially 
for embedded real-time applications. As a demonstration, we 
developed two versions of the same application, first using a CPL 
and then using a common language coupled with an operating 
system. We then compare the respective performances.

\end{abstract}

% ------------------------------------------------------
\tableofcontents

% ------------------------------------------------------
\chapter{Introduction}
\label{sec:intro}

Software development for industrial applications is usually done 
using a high level programming language and an operating system 
for execution control. The operating system offers services that 
can be used to implement an application on a computer.

Coupling high level language programming with a multi-task 
operating system provides comfortable means for application 
design and implementation, so that it seems difficult to do 
without. This pair enables you to produce applications largely 
independent from the hardware, maintenance and evolution of the 
software is made easier.

A family of high level programming languages - called CPLs for 
"Compiled Parallelism Languages" - that include operating system 
features (such as tasks definition, task synchronizations and 
communications between tasks) provides the same advantages. It is 
admitted, from a theoretical standpoint, see~\cite{Halbwachs:91}, that 
these languages bring additional advantages in terms of 
portability, formal modeling and execution efficiency. This work 
is particularly focused on execution efficiency.

We intend to show that CPLs in practice really have significant 
advantages over the classical combination of tools and that using 
CPLs enables you to devise better solutions to various problems.

In chapter~\ref{sec:related} we describe various traditional tools 
and operating systems used to build applications, specially in 
the real-time and embedded domain. We will point out the main 
characteristics of common applications and how classical tools 
deal with their constraints.

In chapter~\ref{sec:cp}  we introduce CPLs by describing several of 
the existing languages and we show their advantages in 
implementing embedded real-time applications. Then we present our 
CPL - MultitaskC - which is more particularly oriented towards 
high performance.

In the last chapter~\ref{sec:evaluation} we describe an embedded 
real-time application, chosen because it is representative in 
terms of constraints. We implement two versions of this 
application : one using an operating system and a 
high level programming language and the other using MultitaskC. 
Then we compare representative results in terms of time and 
memory space performances.

Finally, chapter~\ref{sec:conclusion} we conclude and suggest 
directions for further research.

% ------------------------------------------------------
\chapter{Several development solutions}
\label{sec:related}

There are several methods to program applications. Our purpose in 
this section is to list some of the most representative among 
those used for mono-processor or few-processors machines. 
Applications are often developed using a cross compiler with 
binary codes being loaded on the target system via a suitable 
media such an USB link or a JTAG. Some real-time operating 
systems such as LynxOS make the development environment directly 
available on the target system.

We are now going to describe a few development environments. 
These are usable whether you use cross-compilation or 
target-compilation.

\section{Main characteristics of embedded real-time applications}

Data acquisition applications as described in~\cite{Delchini:95} and 
motor-control applications that will be described here later 
share a set of characteristics. The main ones we will consider 
are :

\begin{itemize}

\item {\em They react to one or more input signals (event-triggered)}. An 
  application which is able to process an event without delaying 
  the processing of further incoming events, is called a 
  real-time application. In this case the response of the 
  application can be considered synchronous with stimulus 
  occurrence.

\item {\em They ensure precise clock-triggered processing}. Very often 
  some processes must be executed at a precise rate. For example, 
  for electric motor control, you usually have to execute a 
  current-control loop at a very high rate (typically 10 us cycle 
  time). The rate's stability is very important for the overall 
  system's stability. It may also be important to control 
  precisely the delay between two separate clock-triggered 
  processes.

\item {\em They fulfill non-triggered servicing}. Most applications need 
  the possibility to specify whether a particular process needs 
  to be executed as fast as possible or only when 
  there is time to do it.

\item {\em They support concurrent processing}. Applications often consist 
  of several more or less independent tasks communicating via 
  standard mechanisms of mutual exclusion and synchronization. 
  For example, in a data acquisition application one task will do 
  data readout and another one data transfers. This way 
  independent hardwares can be operated simultaneously and the 
  total dead time is reduced (provided asynchronous interfaces 
  are available). For maximal efficiency the programmer should be 
  able to specify each task individually as non-triggered, 
  clock-triggered or event-triggered.

\item {\em They interface with specific hardwares}. Real-time systems for 
  apparatus control deal with varied hardwares whose 
  characteristics have profound impact on the software. 
  Interfaces between the processor and the hardware thus are 
  usually quite different from one platform to the other, and it 
  is very important that the programming environment makes it 
  easy to develop and use specific hardware interfaces.

\end{itemize}

\section{A standalone programming language}

This approach is the oldest one and many embedded applications 
have been programmed this way. It consists in using a high level 
programming language, possibly complemented with assembler parts, 
to write a code that is executed independently of any operating 
system. The programmer is in charge of everything and cannot use 
any of the complex objects devised for application design and 
structuring (tasks, means of communication and synchronization, 
...). The execution environment boils down to hardware and a 
possible debugger during the development phase.

The main point here is choosing the right programming language. A 
standard high level programming language with many useful 
libraries, including if possible some libraries of basic 
operating system features, can be advantageous. However the 
choice may be limited by the availability of compilers for the 
hardware target. Whenever possible, a widely used standardized 
programming language makes a good choice.

Several languages meet these criteria. We will retain C 
programming language, as described in [KR:88] and its object 
oriented variant C++. Note that when using C++ the programmer 
should pay attention to the resources (such as memory) required 
by each objects, especially for constructors and destructors, as 
the underlying management of these resources can introduce an 
important overhead during execution. Actually all objects should 
be declared as static memory and no dynamic allocation should be 
used, at least after initializations. In a real-time application 
these hidden costs can become hazardous, as it is difficult to 
control them precisely enough to be sure that the application's 
constraints will always be respected.

Originally, the C programming language has been designed for 
operating system development. It is fit for low level 
programming, and this why it makes a good choice for people 
interested as we are in industrial embedded real-time 
applications. The second important advantage is that C compilers 
are available for most processors. The well known "gcc" compiler 
from the GNU project for example can deal with most common 
hardware (moreover, it has options to be ANSI-standard 
compliant). This compiler suite also offers C++ support, together 
with other popular languages such as FORTRAN. A standardized 
debugging interface is provided with "gdb", as well as binary 
tools for specific linking. Like all modern compilers it 
implements very high level optimization techniques such as 
inter-procedural analysis, for example.

Many real-time embedded applications have been programmed without 
using an operating system for execution control. It is a long 
standing programming method. Applications written this way 
consist mainly of interrupt-routines dedicated to event-triggered 
processing. For clock-triggered processing, an interrupt can be 
generated by a timer at the desired rate . Other interrupts can 
be used to indicate the availability of input-data or to start a 
data transfer.

High-level programming languages such as C don't dispense from 
writing some parts of the application in assembly language. 
Typically because some instructions or registers are 
processor-specific and cannot be dealt with by high-level 
languages. Unless coding and precise structuring rules are 
defined, the programmer is allowed to do whatever he wants. As a 
consequence diverging coding styles may become a problem, 
particularly in case of large team work, and more generally as 
far as maintenance is concerned. One can avoid such drawbacks 
however, since these simple programming tools make it easy to 
structure the application as a set of finite-state automata for 
example, or even Petri nets in case of multi-core architectures.

Such an approach, though quite basic, is widely used. The only 
point we'll keep in mind is the use of the C programming 
language. We are now going to give an overview of how to combine 
this programming language with an operating system.


\section{A programming language with an operating system}

Such a development environment has been commonly used for many 
years in the real-time community. More and more micro-controllers 
are able to run operating systems as the resources needed to do 
so are more and more available. Let's have a look at some 
operating systems and other similar tools. In each case we 
consider the programming language used is C. 

They were chosen because they are representative of what is 
commonly used in the industry. PharOS comes with its own 
programming language, but the language is very similar to C. We 
mention OS9 which is an early real-time operating system but we'd 
like to focus on more recent ones. Note that VRTX is the ancestor 
of VxWorks.

We will also do an overview of the ADA programming language 
explain below why we consider it as a language coupled with an 
operating system. In the section dealing with development 
environments evaluation we consider PharOS with the PsyC 
programming language. In~\cite{Delchini:95} we use VRTX and LynxOS 
with the C programming language. Also we have used C and C++ with 
VxWorks in an other evaluation that will be described shortly.

\subsection{VRTX}

All the information we present here is taken from~\cite{VRTX:90} and~\cite{AMD29000:91}.

VRTX ("Versatile Real Time eXecutive") is a real-time fixed 
priority operating system designed for industrial applications 
development with constraints in terms of performance and 
reactivity. It consists of a customizable kernel that can be 
configured to fit exactly the application's needs. We've did all 
the evaluation tests on a version customized for the AMD29000 
micro-processor but there are other version for common targets 
such MC68000 and others.

Developments were done in C using a cross compiler on a PC. 
Application code is downloaded onto the target via an industrial 
bus called VME. An on-target environment for VRTX is also 
available but we didn't use it. The main characteristics of this 
operating system are : mono-user system with multi-tasks 
capability and a fixed priority scheduler. There are three 
execution levels : supervisor mode, user mode and a mode for 
interrupt routines. The version used in~\cite{Delchini:95} of the 
operating system shares the physical address space for data and 
instructions with user tasks. There is neither memory 
segmentation or virtualization nor memory protection via a MMU.

Configuring VRTX to support paginated memory is possible but this 
is not the option by default. For performance considerations we 
preferred to have simple memory management because pages' tables 
handling during contexts commutations introduces an additional 
system latency. This feature however could be interesting for 
safety reasons as we will see with PharOS and LynxOS. System code 
can be flashed into ROM, which is a common usage for embedded 
applications. Note that this has an impact on performances 
because ROM access time is usually really slower than RAM one.

System calls do not conform to any standard but the C compiler we 
use conforms to ANSI C. The kernel is available as an assembly 
source file describing in a big data area the binary code of VRTX 
. This can be used to link the operating system with additional
software such as startup code relative to the mother-board support
(board support package) and drivers. The board specific software
has to be written by users and this allows to configure the kernel
for a particular use. Several initialization informations must be
defined, such as maximum number of tasks, stack-sizes, etc. Writing
this piece of software is not easy and when it is done, you still have to 
develop the entire application.

VRTX can be configured with different hooks to handle specific 
processing related to tasks management. There is one hook for 
task creation, one for tasks commutation and one for task 
deletion. In our implementation we also used hooks to manage the 
AMD29050 floating point co-processor, in order to backup and 
restore the FPU registers on context switching. Other hooks can 
be used to handle the MMU/MPU contexts during the same task 
operations, if memory protection and segmentation is needed. For 
performances issues, in our evaluation we dismissed MMU/MPU 
management but used the FPU for motor control.

The operating system handles dynamic memory allocation and basic 
input/output functions on a serial line if the related driver is 
included in the configuration software. Standard C library 
functions such as printf/scanf and malloc/free must be also 
implemented by the user.

In the AMD29000 version, several processor registers are reserved 
for the operating system. This feature isn't detrimental because 
the hardware implements a big amount of registers (192) but 
improves execution efficiency because it reduces the number of 
registers to save/restore during context switching. As regards 
the number of registers dedicated to each task, note that 
although a big number of registers helps the C compiler do 
optimizations it implies a significant cost during tasks 
commutations. Since there are much less registers in the MC68000 
processor architecture, no registers are reserved for system use 
in the MC68000 version of VRTX. We are now going to review some 
features of the operating system, been understood that a lot of 
the preceding characteristics can be found in other real-time 
operating systems.

\paragraph{Tasks.} As we said before, VRTX is a multi-task operating system. A VRTX 
task corresponds to the execution of binary instructions by the 
processor. The operating system handles time sharing between all 
ready tasks in a given priority. A task can create other tasks 
but there are no ancestor or sibling relations between them. A 
task can get no information upon it's ancestor. Tasks are created 
and deleted without any incidence except context switching with 
respect to priorities. There is no implicit meeting between a task
and the task which created it at the end of execution, for example.

Each task is allotted a priority when created. Priorities enables 
the system to determine which task among the set of ready tasks 
will run at any given time on the processor (scheduling). VRTX is 
using a fixed-priorities scheme, unlike time-sharing multi-tasks 
operating systems such as UNIX. In these systems, the priority of 
any given task varies with time. An intensive time-consuming 
task's priority will be lowered to avoid processor starvation for 
the other tasks. Later its priority will be raised again to its 
initial value when the system considers it should anew be 
allocated processor time. This is called a "fair operating system".
Like many fixed-priority operating systems, VRTX is an unfair 
one : the highest priority task may get hundred per cent of 
processor time.

There are two scheduling modes in VRTX. In the first mode, all 
the ready tasks of highest priority are executed in a round robin 
manner, each task running for a maximum time called a quantum 
(typically 10 milliseconds). The operating system gets processor 
control at the end of each quantum via a hardware clock interrupt 
(timer). In this mode the system is fair for a given priority 
level. In the second mode, the quantum is infinite so that the 
running task must relinquish the processor (either explicitly or 
via a system call) to allow executing an other task.

We have mentioned previously that in our VRTX configuration there 
is only one address space shared by user tasks and system. An 
other feature to be mentioned is that a private stack is 
allocated to each task so that it is possible to implement a 
function-call mechanism with parameter passing or local dynamic 
allocation of variables in a function. However there is no 
stack-memory protection in our configuration.

Let's review shortly now basic tasks-related functions. {\em Create} : creates the task, specifying priority, entry point, 
and execution mode (supervisor or user). {\em Delete} : immediate cancellation of a 
task (a task can also terminate itself). {\em Suspend} : the task goes 
to the non-ready queue (a task can also suspend itself). {\em Resume} : once suspended, a task 
can be resumed by another one. {\em Priority} : changes the priority of the current task. {\em Information} : recovers 
the address of the system internal task-control block containing relevant informations.

VRTX is an unfair operating system because it is designed for 
real-time application control. A high-priority task can 
monopolize the processor, releasing it only when it needs to do 
so. Other system calls allow a task to lock context-switching or 
to define windows of execution with limited quantum. Once the 
operating system is locked, it behaves as a mono-task system with 
only one task running. The operating system can also be unlocked, 
of course. Finally, there is one task of lowest priority in the 
operating system - the idle task - that is executed when no other 
task is ready to run. The default idle task can be replaced at 
configuration time by a user idle task which can be used to 
implement a low priority activity.

\paragraph{Communications between tasks.} VRTX provides four types of objects to deal with communications 
between tasks. These standard objects that can be found in other 
operating systems allow the user to implement simple 
(producer/consumer for example) or complex communication schemes. 
Communication tools are listed below (note that it possible to 
specify a time-out for blocking system-calls). {\em Mailbox} : a 
mailbox is a 32-bits memory where the user can post a value (non 
blocking action), an error is raised if the mailbox is full. 
Symmetrically, to pend on a mailbox is a blocking action if the 
mailbox is empty. If more than one task is pending, the 
highest-priority task or next task in the same priority level 
will be served. {\em FIFO} :
a FIFO is a set of 32-bits values. The 
user specifies the FIFO size when he creates the FIFO, and 
similarly to a mailbox, posting to a FIFO is not blocking while 
pending on a FIFO is blocking if it is empty (but it is possible 
to check the case before pending). A user can ask for the current 
FIFO size and jam a new element into it. There is a system call 
to get a copy of the next element in the FIFO without removing 
it. Two policies for reactivating a task pending on a FIFO are 
available : either the systems activates the task of highest 
priority (as for mailboxes) or it activates all the tasks waiting 
for that FIFO in sequential order of arrival. At FIFO creation, 
the user can specify the policy to be applied. {\em Flags set} : a 
flag set is made of 32 flags, each one bit in a 32-bits word. In 
order to check the flag, the user gives a mask and a parameter 
indicating if he requires an "and" or "or" wait condition. If the 
condition is satisfied, checking the flag is not blocking. 
Checking a flag never changes its value, there is a system call 
to do that. More than one task can be wakened on flag condition 
at the same time, task-execution being done according to 
priorities. {\em Counting semaphores} : a counting semaphore can have 
a value between 0 and 65535. It is a well-known concept 
introduced by Dijkstra to handle mutual exclusion while accessing 
to a shared resource. The user can increment a counting semaphore 
(non blocking action). He can also decrement it : if the value is 
zero, the calling task is blocked. Task-wakening options are the 
same as for FIFOs.

Lastly, VRTX provides a mechanism called "shared memory"
data between tasks. This is possible because, as was said before, 
there is only one address space in the system. Combined use of 
semaphores and shared memory allows the user to implement 
communications schemes as complex as necessary.

\paragraph{Interrupts.} Some key point concerning interrupts in general. The concept of 
processor interruption by peripherals is a basic mechanism for 
operating system design. System calls are usually implemented 
through software interrupts that are similar to 
hardware-generated ones, but are raised by a specific processor 
instruction. When an interrupt occurs, the processor execution 
flow is restarted at an address called an interrupt-service 
routine address. Execution mode is automatically set to 
privileged and further interrupts are disabled. Several registers 
are saved either on a stack or into additional special registers 
that can be afterwards saved in memory if necessary.

This mode of operation allows the operating system to manage all 
resources coherently : save/restore task-contexts, handle 
communications, etc. all this can be done without interruption 
during critical phases. The operating system also controls 
peripherals via interrupts, such as for instance a peripheral 
telling the processor that data is available by raising an 
interrupt. In real-time programming interrupts are very often 
used to trigger a process, for example a data-acquisition one. 
Interrupts can also occur at specific rates for clock-triggered 
actions.

As regards interrupts, two different operating-systems behaviors 
are possible. Interrupts may themselves be executed atomically while 
the operating system is running in privileged mode and system 
calls are then atomic. They execute without being interrupted ; 
if an interrupt occurs, the corresponding handler execution is 
delayed. These operating systems are said to be non-preemptive in 
privileged mode. This behavior has a detrimental impact on 
operating system latency, and is not suitable for real-time 
applications. VRTX is preemptive in privileged mode, so operating 
system implementation is more difficult but interrupt latency is 
better. VRTX allows interrupts as much as possible, so delays due 
to handlers are minimized.

In VRTX, a special stack to backup registers during 
interrupt-routine prologue is available. Such a routine can be 
activated by hardware without indirection by the interrupt 
driver, as it is the case in many real-time operating systems. 
This reduces interrupt latency. An interrupt-routine can 
communicate with the task level according to fixed priorities, 
for example by posting a value in a mailbox to trigger the 
task-level process. However, such a mechanism must comply to a 
system protocol : ie, when any any system-call is used inside an 
interrupt-routine, the interrupt-routine entry and exit must be 
signaled to the system. At the end of interrupt routine, control 
is given back to the operating system in supervisor mode so that 
rescheduling can take place. This protocol is time-consuming but 
actually less than the protocol implying interrupt drivers in 
many other operating systems. Note also that for some specific 
applications, it is possible to do without this protocol in order 
to get better interrupt latency.

We are now going to describe an other operating system dedicated 
to real-time applications.

\subsection{LynxOS and POSIX real-time extensions}

Before describing the Lynx operating system, we are going to give 
some general information about the POSIX standard ("Portable 
Operating System Interface", X standing for similarities with 
UNIX). Information listed here is essentially taken from~\cite{LYNX}, \cite{JMR:93} and \cite{JMR:94}.

POSIX reference number is 1003 in the IEEE directory ("Institute 
of Electrical and Electronics Engineers") which intends to 
establish industrial standards. POSIX is also referenced as the 
international standard ISO 9945 ("International Standards 
Organization"). An ISO standard contains the description of all 
application programming interfaces for the operating systems that 
implement it. These interfaces specify data types, types names, 
functions names and semantic. As POSIX-compliant applications can 
be considered operating-system independent, and as C is also a 
standardized system-independent programming language, we'll 
assume POSIX and C to be our reference framework. Most UNIX 
implementations implement the POSIX interface, and this is also 
true for VxWorks and LynxOS. There are several subsections in 
POSIX, the basic features are in 1003.1, real-time extensions are 
in 1003.4 and real-time threads extensions in 1003.4a.

The LynxOS operating system is aimed to be POSIX and ANSI-C 
conforming, which is a interesting approach. The ANSI-C part is 
based on GNU compilers and libraries. As regards the operating 
system interface, LynxOS includes all POSIX (including real-time 
extensions). Besides, LynxOS is compatible with the two main UNIX 
flavors : "System V" and BSD. LynxOS is a multi-tasks and 
multi-users operating system, with execution levels similar to 
VRTX but implementing in addition a notion of user similar to 
UNIX. LynxOS is available for several targets such as IBM PC, 
Motorola VME boards based on MC68000 or PowerPC, SUN Spark 
workstations, several HP embedded platforms, IBM PowerPC based 
computers, etc. The operating system is delivered in binary code 
plus a collection of object files allowing the user to customize 
it. The complete development environment is located directly on 
the target system.

The kernel size may range from 200Kb to more than 1000Kb with all 
possible extensions. It can be stored in ROM with a minimal ROM 
file system, and it can also be booted via a network. These two 
possibilities allow easy deployment of applications on minimal 
hardware configurations, as well as easy implementations of the 
same software on multiple hardware targets.

Although LynxOS integrates a lot of UNIX inspired standards, its 
kernel is really different because it is designed for real-time 
applications. Its kernel design is very similar to VRTX, with the 
difference that its development environment is located on the 
target. Our evaluation system is based on a MC68040 VME board 
with LynxOS installed on a local hard disk.

Memory management is paginated and can also be virtual. Each 
process has its own address space which is private. Like any user 
process, the kernel has its own address space but kernel pages 
are always resident in core memory. When virtual memory 
management is not enabled, all process pages are also resident in 
core memory. In order to have virtual memory enabled, a swap area 
must be defined on hard disk to store the pages that temporally 
cannot stay in core memory. As users processes and kernel cannot 
access each other address spaces, such a memory segmentation is a 
feature of safety.


\paragraph{Processes and activities.} The basic execution entity of LynxOS and UNIX are the same : the process. However, when 
the process of UNIX has a time-dependent priority, the process of LynxOS has a fixed priority. Its scheduling of processes is 
based on queues of different priorities, with a round-robin allocation of quantums inside a given priority queue as in VRTX. 
The quantum can differ from one priority level to the other, so as regards scheduling LynxOS is fair only for tasks with the same 
priority (this is close to real-time processes handling in "System V.4" version of UNIX).

When virtual memory is used, LynxOS swapping mechanism is handled by a system process which has itself a priority
as any other process. This means that the relative priorities of processes have an influence on memory management. A process with 
priority higher than the memory-management process will never be swapped when a process with a lower priority can be swapped. 
In addition, a process can lock pages in core memory without modifying its priority. This feature must be used with
caution because it may lead to memory starvation. A LynxOS process can be seen as a unit of resources encapsulation, such
as file descriptors, memory space, etc. The execution of a program is implemented as a "thread". Any LynxOS process consists of 
one or more threads that are running in the process address space.

As for VRTX tasks, each thread owns a private stack for function calls and automatic variables allocation. It is given a fixed 
priority when created. A LynxOS process always has an initial thread, a specific feature not specified in POSIX recommendations.
The ensemble of a LynxOS process with its initial thread is equivalent to a UNIX original process. As any thread can
monopolize the processor, LynxOS scheduling is not necessarily fair. LynxOS threads implementation conforms to POSIX, for more 
informations about POSIX threads see~\cite{JMR:94}.

We are now going to list the main operations available for threads control under POSIX. {\em Create} : thread creation. 
The operating system assigns a unique thread identifier that is usable for further operations. Some parameters set priority, stack size,
scheduling type (FIFO, ROUND ROBIN, DEFAULT), etc. There are defaults values for these attributes. {\em Self termination} :
a thread triggers its own termination, returning an exit code that can be an address in the stack. As stack memory is not released after 
termination, this data area remains accessible by other threads. {\em Termination by another thread} : a special exit code is returned.
{\em Detach } : releases all the resources of a terminated thread. Several scheduling algorithms are possible : DEFAULT scheduling 
combines a fixed priority and a specific quantum for each priority level, ROUND ROBIN and FIFO scheduling are similar except that there 
is only one global quantum for all priority levels, finite in the first case and infinite in the second.

\paragraph{Communications.} LynxOS implements two sets of communication tools, one for inter-process communications and the other for 
threads. Inter-process communication tools are implemented as a mix of UNIX "System V" and BSD flavors, LynxOS including IPCs ("Inter
Process Communications") and STREAMS from "System V", and UNIX domain sockets from BSD. POSIX standard changed recommendations
in 1003.4 Draft 9 concerning "System V" IPCs. The use of communications tools (semaphores, shared memory, message queues) is made
easier. For threads running in a process, additional simple tools have been designed so that the user can implement more efficiently 
complex communication schemes.

Besides shared memory inside a process address space, LynxOS implements POSIX tools. {\em Rendezvous} : a thread can wait
for the termination of an other thread via a rendezvous. When the termination occurs, the waiting thread gets the exit code of the 
terminated thread. {\em Mutexes} : a mutex is similar to a binary semaphore but only the thread owning the mutex can release it. 
A thread is blocked when it tries to get a mutex that has been taken by another thread. This is used when mutual exclusion between threads 
is needed, for instance when shared resources must be accessed in a critical section. {\em Condition variables} : combined to mutexes, 
it allows complex synchronizations between threads. Once it owns a mutex, a thread can wait on a condition variable. The mutex is released 
while the condition is false. When a thread signals the condition variable, the mutex is automatically taken and the waiting thread is 
awakened. It is possible to wakeup selectively one thread among a set of threads waiting on the same condition with the same mutex, as well 
as all the waiting threads at the same time. A thread can also use inter-process communication tools, and LynxOS includes tools that
are not required by POSIX such as fast binary and counting-semaphores or shared memory.

Software signals similar to the UNIX ones can be sent to processes or threads. A signal sent to a process is randomly taken by one of the 
process's threads which accept that signal. If none of them accepts the signal, it stays pending for a thread to accept it. A signal can also 
be sent to a particular thread of a process.

\paragraph{Interrupts.} LynxOS does not allow an application to catch a hardware interrupt by itself. To LynxOS designers
this direct mode would have lead to the implementation by users of services that should be handled by the operating system. We have seen
in VRTX that this direct mode is better as regards interrupt latency but implies writing more code to be implemented. With LynxOS, the only 
way to handle interrupts is to use system drivers. Only drivers can attach a hardware interrupt to a handler functions. Usually,
interrupt routines are as short as possible, their main duty being to trigger one or more system threads to handle more time-consuming tasks. 
These threads are executed in kernel space. As they are themselves scheduled, the user can manage interrupts priorities.

A particular counting-semaphore type is used by an interrupt handler to communicate with the system-threads. There is only one
kernel handler that catches all hardware interrupts, prepare the execution context and then executes a driver handler if installed.
This allows easy system coherency but introduces interrupt latency.

\subsection{The ADA programming language}

All informations listed here come from~\cite{Booch:86} and~\cite{ADA:83}. The ADA programming language is a high level language
that integrates a lot of features. One of the target application field is real-time development and we will describe here ADA
as a programming language that allows to specify applications with asynchronous communicating tasks. There are two standards
describing ADA : ANSI and ISO.

\paragraph{Tasks.} It is possible to declare tasks in an ADA program. Execution of them is either controlled by a kernel linked
with the application executable or by a host operating system not included in the ADA compiler. In the first implementation,
the ADA compiler builds an executable by sources file compilation and link with a micro-kernel. In the second one, the compiler
generate code for task creation in the host operating system.

The user can use the "PRIORITY" pragma to set priorities to ADA tasks. These priorities can have different significations
depending on the host operating system. The scheduling algorithm is not specified in ADA standard and neither the way to schedule
two tasks at same priority level. An ADA program is made of tasks similar to UNIX processes, LynxOS threads or VRTX tasks, this
is why we consider ADA as an operating system coupled to a high level programming language.

\paragraph{Communications in ADA.} The language includes mainly one communication tool : the binary rendezvous with data passing.
An ADA task can accept a rendezvous with another task via the "accept" operator. The rendezvous is realized when at least two
tasks are waiting on the "accept" operator, if there are more than two tasks, priorities may influence the two tasks selection.
A rendezvous is always blocking when only one task is trying to accept it. It is optionally possible to transfer data when a
rendezvous is realized, similarly to a parameter passing for a function call.

It is possible with the "select" operator to wait for more than one rendezvous according to the state of local variables in a task.
Only one rendezvous will be realized and if none is possible, the user can specify a default action to be taken. If more than
one rendezvous is possible, tasks are selected to realize the rendezvous according to priorities, the highest priority will be
used. If all the tasks are in the same priority level, it is the scheduling implementation of the underlying operating system
that will decide. It is also possible to declare shared memory segments in ADA that can be combined with rendezvous to
implement complex communication schemes. But usually, rendezvous with data exchange is enough.

\paragraph{Interrupts.} It is possible to associate a hardware interrupt to an ADA rendezvous. Only one task then is allowed
to wait for this rendezvous to occur. When the interrupt is taken, the corresponding rendezvous is realized and the waiting
task is awaken. The task can get some read-only data when the rendezvous occur. An interrupt can be seen as the highest
priority task always greater than real tasks. In this way, the rendezvous associated to an interrupt is at highest priority.
The scheduling algorithm of the underlying operating system is not involved in interrupts rendezvous. Interrupt handling
implementation for the host operating system is not specified in the language standard. It can use a direct handler without
an interrupt driver or another implementation.

\subsection{VxWorks}

VxWorks is a multi-task operating system very similar to VRTX, obviously because VRTX was the kernel included in VxWorks in
initial versions (VxWorks stands for "VRTX Works"). So tasks management is almost the same and communication tools also.
VxWorks comes with a very user friendly configuration tool for board support package and kernel features. In version 5, it is
called "Tornado" and in version 6 is is based on "Eclipse" and called "Workbench". This configuration tool is also an
Integrated Development Environment for the programmer with cross compilation. The user can use C programming language but also
C++ and other languages that the compiler can handle (gcc based).

Note that from version 6, VxWorks kernel introduced Real Time Processes that is an implementation of memory segmentation
similar to POSIX ones. This can be useful for applications that need safety features. Interrupts are handled the same way
than LynxOS with an interrupt driver. The operating system is available on main hardware such as IA32 and PowerPC. Note that
in the development phase of an application, the user can include a shell with a minimal C interpreter that allow to load
binaries and call functions with parameters from the shell. Once the application is ready, the user can remove the shell from
VxWorks configuration. VxWorks also offers a very good instrumentation tool with possible kernel and application instrumentation,
it is called WindView and can really help the user to improve performances and robustness of the application.

\subsection{PharOS}

More informations about PharOS can be found in~\cite{OASIS:98}. The PharOS operating system is being developed for safety critical
applications. It is mainly composed of the PsyC programming language and a real-time kernel. The PsyC language is very similar to C and
includes the possibility to describe tasks and also provides communication tool as "Temporal variables" and "Message queues".
The particularity of this operating system is that it is time triggered. The user can define clocks in PsyC sources and then all
processes must occur in temporal windows defined in tasks. All treatment must end before a deadline and scheduling is made according
to Earliest Deadline First algorithm.

Interrupts can be handled with PharOS but are time tagged when they occurs and associated with a process in time triggered
space. Thus event triggered processing is converted to time triggered ones. There is an automatic memory segmentation using MMU
or MPU hardware to do spatial partitioning. This is similar to the partitioning done in VxWorks or LynxOS. Spatial or temporal
partitioning violations can be handled by the operating system with task groups. If groups are defined in an application, when
a task of that group has a failure, the entire group is restarted at entry point with an indication that it is a failure
recovery startup. PharOS is available for IA32, PowerPC, MC68000 and ARM targets.

\subsection{Features summary}

We are going here to summary the characteristics that seem useful in the development and execution tools we have described. Except
for the standalone programming language, they all offer the possibility to specify an application structured as several tasks
that can communicate and synchronize. This is a very common design scheme for real-time applications so it is almost necessary
to have it for a good development tool.

The operating systems we have described are preemptive in privileged mode except PharOS in micro kernel mode. As almost all
real-time applications are using interrupts, this characteristic is mandatory to reduce system latency.

With or without cross compilation, it is possible to program in a high level programming language and debugging tools. Assembly
language can be can in special cases to access particular registers or instructions and its usage can be very minimized to ease
portability. Note that operating system portability is not simple as it is hardware dependent for various features even if
it is coded in a standardized high level language.

An answer to this portability problem that brings several other advantages can be a programming language family we are going to
describe.

\section{Synchronous programming languages}

Synchronous programming languages with compiled parallelism appeared in early 80 for Esterel, implementing the synchronous product
of deterministic finite state machines defined by~\cite{ArnoldNivat:82}. We will introduce several of these languages starting with
Esterel which is one of the first. Then after the description, we will compare the OS/language approach to CPLs. Following this
comparison, we will introduce the language we have defined which intend to give to the programmer the same features than an
operating system coupled to a high level language but with compiled parallelism for efficiency. We will use our language to represent
CPLs in our evaluations as it implements the same paradigm. Informations here comes from~\cite{Halbwachs:91} regarding Esterel, Lustre
and Argos.

\subsection{Esterel}

Esterel was developed by a common team of the "Centre de Mathematiques Appliquees" of "Ecole des mines" and of INRIA ("Institut
National de Recherche en Informatique et Automatique"). It's a programming language that mainly allow the user to specify
behaviors and validate them, this is why it does not offer complex data structures for example that you can find in common
languages. It is possible with Esterel to specify an application as a set of communicating tasks.

An Esterel program describes a reactive system, this means a system which react to external stimuli and as a reaction emit other
stimuli. The result of an Esterel compilation is a finite state machine coded in a high level language call host language (C,ADA,...).
Automata transitions are implemented in a host language engine. This engine execution after compilation for the hardware target
realize the behavior described in the original Esterel code. This execution consists in chaining transitions by the automata engine.
The use of a host language allow easier portability.

The reactive system does not usually implement the entire application. An application is made of several reactive systems that are
activated by a main program by triggering transitions in each engines while handling other functions (initializations,interrupts,...).
The language syntax is very close to PASCAL and ADA even if the underlying execution model is completely different. There is
a C front end for Esterel called "Reactive C" that may be more easy to use.

It is possible in an Esterel program to define local variables, to call host language functions, to define instructions sequences,
to have control statements such as loops and tests and common computing expressions with usual operators. The user can define modules
that can be included in further programs so applications can be structured. There are no global variables in Esterel.

\paragraph{Tasks and communications.} The "\textbar\textbar" operator allows the user to specify independent tasks. Communications between
tasks is made by signals as for external communications. It is the only communication tool in Esterel and for example there is no
shared memory. A signal can be externally or internally raised for the reactive system. It can be used for internal communications
only, a tasks can raise a signal ("emit() operator") and then all pending tasks (with "do halt watching signal") can be awaken.
Waiting on a signal can be blocking but a task can just test the state of a signal (with "present"). It is possible to
associate a value to a signal and then emission can modify it. A task can get the signal value when taking it in account.
This is also true for external signals so values can be transmitted with the outside world. The interface between the reactive
system and the outside world is simple. Signal emission from the system to the outside world is translated into a function call
in the host language so the developer must implement a function for each output signal. For each input signal, the Esterel
compiler builds an input function that should be called by the user to raise the signal internally. A signal can be both for
input and output, in this case there are an input and an output function.

The execution of a single reactive system can be seen like this : a main program raises signals to the reactive system by
calling associated functions then call the automata engine for the next transition that will trigger signals emissions. This
execution scheme can be quite efficient because there is only one context to handle for all reactive systems.

\paragraph{Parallelism.} Tasks handling is completely different in Esterel than for the operating systems we've described before.
The parallelism is compiled, so to say, if several tasks are described in an Esterel program, after compilation there is only
one task which realize the execution of all the original tasks. So it is not possible to create or destroy tasks dynamically
in Esterel. A multi-task operating system simulates parallel execution of tasks with dynamic runtime scheduling but in languages
such as Esterel, scheduling is statically evaluated at compile time. Each Esterel task is considered as a finite state machine
and parallelism is compiled by a synchronous product of all the input state machines giving in turn a state machine which
when executed simulates the concurrent execution of operand state machines. We will also use this principle for our programming
language. Finite state machines are well known mathematical objects that can be used to formalize behaviors. So the synchronous
product gives two things : a formal model of program execution that can be used for properties proof and parallelism compilation.

\paragraph{Interrupts.} As for operating systems it is important to be able to handle interrupts with Esterel. Obviously, a signal
can be associated to an interrupt by calling the signal input function in the interrupt handler. The signal occurrence is then
taken in account at execution level. This is quite similar to operating systems way to handle interrupts but there is one
significant difference because here interrupts can be allowed at any time (except during interrupt handling itself) so latency
can be minimized. This is a important point for efficiency in real-time applications.

\subsection{Other synchronous languages}

A lot of other synchronous languages have been implemented and we are going to describe two of them rapidly.

\paragraph{Lustre.} As for Esterel, the result of a Lustre compilation is a finite state machine coded in a high level host
programming language. Lustre is a declarative language and a Lustre program is made of operators declarations (nodes with inputs
and outputs). Operators can be combined for build more complex ones. Lustre is based on the data-flow paradigm. This means
that a Lustre program is an operator network. For example, an operator can have two inputs and one output that can be connected
to other operators and so on. An operator in the network can be executed if all its inputs are valid. This execution model
integrates parallelism if all ready operators at a time can be executed. A Lustre program execution is cadenced by at least one
clock. Each clock tick triggers the execution of all ready operators. The compilation of operators is made by a synchronous
product as in Esterel.

\paragraph{Argos.} Argos is almost equivalent to Esterel. The main difference is that it implements a graphical syntax. An
Argos program is made of graphical finite state machines that can be combined with a parallel operator. The result of automata
combination is an automata that can be in turn combined with others. An Argos compilation produce with a synchronous
product an automata coded in a host language.

\section{Comparison of CPLs and "language/OS"}

All programming solutions we have described allows the user to build real-time applications using classical features
(communicating tasks). We are now going to compare CPLs with OS/language approach. After this comparison we will try
to verify some points in the evaluation section.

\subsection{CPLs drawbacks}

A limitation of CPLs is given by the synchronous product which is used to compile parallelism. The size of the result
in terms of states and transitions number is potentially exponential of the operands sizes. The external stimuli can
also increase this complexity. This limitation is theoretical and in practice it is rarely reached. However this limitation
does not exists with an operating system. We will introduce later the possibility to avoid this exponential complexity
with a particular way to implement the synchronous product in our language. In the same way, the space complexity of a
synchronous product is potentially exponential. This is also irrelevant for an operating system even if it also
consumes memory space for its own data that is not consumed with a CPL.

Another point is that an operating system allows dynamic creating/deletion of tasks. This is not possible for a CPL
based on compile time synchronous product where tasks and communication tools must described statically. We will also
introduce later the runtime synchronous product we have implemented in our language and that can be a solution
to handle dynamic objects in a CPL.

\subsection{CPLs advantages}

CPLs allows the user to write programs completely independent from an operating system. An application written with a CPL
has a better portability and will only depend eventually on a standard C library that is always available with today
compilers and also needed with an operating system. The use of a host programming language is also better for portability.
Even when you have to upgrade existing hardware, for example to support an arithmetic co-processor, it's easier with
a CPL. With an operating system, the context of each task may be impacted to save/restore FPU context. This has an
impact on performance and need an operating system modification. With a CPL the same integration is just handled
by the host language compiler, in a case the compiler gives FPU emulation and in the other generates real FPU instructions.

A CPL allow the user to specify an application using standard features also given by operating systems such as tasks
and communication tools. So the programmer doesn't have to integrates new paradigms. We will see later that in the same
time, CPLs brings more performance than operating systems with the limit given by hardware only. Another difference
is about interrupts which is very important for real-time applications. In a real-time operating system, there are
a lot of sequences executed with interrupts disabled. This is absolutely necessary to provide system coherency
during scheduling for example where there are some linked lists to be managed. Some system calls must also be
atomically executed from task level. Even if the operating system is preemptive in kernel mode, it is not possible
to be always the case. With a CPL, it is possible to have interrupts enabled all the time except of course during
the execution of handlers prologues and epilogues. There is no need to have critical section toward interrupts
because there is only two execution levels : application and interrupts, and only one task context.

With an operating system, the communication between interrupts and tasks execution levels has a bigger cost than with CPLs.
With an operating system, signaling a counting semaphore will trigger scheduling and a lot of actions to have the
task level activated for processing. With a CPL, the interrupt routine can be reduced to a simple counter
incrementation without critical section just to indicate task level that some processing must start. This also
involves that with a CPL, interrupt routines can be as short as possible so there is more time for task level.

In multi-task operating systems, each task must have a data structure usually called context made of all processor
registers plus a stack and optionally some MMU/MPU descriptors. When a task is chosen by the scheduler, the previous
task context must be saved and the new task one restored, this called context commutation. This operation is
not necessary with compiled parallelism because there is only one context, so no dynamic scheduling and context
commutations. This is also the same for communications and synchronizations, every thing is statically handled
at compile time and there is no cost at runtime. With a CPL, as there is no context commutation cost, it is
possible to tune with a very fine grain processor execution time. So it is possible for a task to release CPU
for very short periods that can be used by other tasks. This is not the case with an operating system.

The result of the synchronous product in the host language gives the final target compiler a vision of all
application tasks. This can allow the target compiler optimizer to do global time and space optimizations that
is not possible with an operating system because tasks code and data are completely separated.

% ------------------------------------------------------
\chapter{Compiled parallelism and MultitaskC}
\label{sec:cp}

\section{The MultitaskC programming language}

MultitaskC syntax is a C language extension, this is the most popular programming language so the choice is obvious
to us. We consider that the reader is familiar with this programming language as described in~\cite{KR:88}.

A first extension of MultitaskC is the possibility to describe independent tasks. We will call task a C statement or
compound statement in parallel with at least another C statement or compound statement. We have implemented the same
paradigm as in other CPLs for parallelism compilation and we are targeting mono-processor architecture. The result of
an MultitaskC compilation is a finite state machine coded in pure C. Execution of this automata is simulating
the parallel execution of all the tasks. Common communication and synchronization tools are implemented : rendezvous
and an implementation of the Esterel synchronous broadcast. There is also the possibility to use share
memory via global variables.

The user has the possibility to declare meetings or rendezvous at the beginning of a function and the scope
of the rendezvous is the entire function. The declaration is following this grammar :

\begin{table}[h!]
\tt
\footnotesize
meeting\_declaration : meeting meeting\_list ';'\\
meeting\_list : 'id' ':' 'constant'\\
meeting\_list : meeting\_list ',' 'id' ':' 'constant'\\
\end{table}

Here are few valid rendezvous declarations, the constant indicates the number of tasks needed for the rendezvous to
occur (respectively 2, 3 and 4) :

\begin{table}[h!]
\tt
\footnotesize
meeting rdv:2;\\
meeting rdv1:3,rdv2:4;\\
\end{table}

All C instructions are included in MultitaskC such as "for()", "while()", "if()" control statements. Expressions for
tests and statements are the same, also all operators and declarations. All expressions are executed atomically and
there is an operator to let a sequence be atomic : "group". For example, the "for()" loop is atomic :

\begin{table}[h!]
\tt
\footnotesize
group \{ for(i=0;i<j;i++) func\_call(i); \}
\end{table}

For more complex instructions the following rules are added to standard C grammar where "id" stands for a block
identifier or a rendezvous :

\begin{table}[h!]
\tt
\footnotesize
statement : conc\_statement\\
statement : sync\_statement\\
statement : block\_statement\\
statement : {\bf break(id);}\\
conc\_statement : {\bf execute} statement {\bf and} statement\\
sync\_statement : {\bf when(id)} statement\\
sync\_statement : {\bf when(id)} statement {\bf else} statement\\
sync\_statement : {\bf join(id)}\\
block\_statement : {\bf block(id)} statement\\
\end{table}

The "join()" and "when()" instructions allows to communicate with rendezvous that must be declared before use.
With this declaration "meeting rdv:3;", tree tasks must have reached the rendezvous to realize it whether with
the "join()" instruction or with the "when()" one. "join()" blocks the task until the rendezvous is realized and
"when()" allows to test a rendezvous status and if it is realized the "then" part is executed, the "else" part
if not. Note that there is an implicit rendezvous at the end of an "execute and" block, the two tasks must
be terminated for the block to terminate.

The "break()" instruction combined with "block()" allows a task to interrupt another one or itself. "block()"
is used to name a statement or compound statement and then when a "break()" with the same name is executed,
all the tasks in a block of that name are resumed to the end of the block.

\section{Compilation}

An MultitaskC compilation of a source code with at least two tasks consist in the translation of the two
tasks in two finite state machines with an intermediate representation. Then a synchronous product
of these two state machines is done by the MultitaskC compiler and C code is generated as a result.
If more than two tasks are present in source code, this is the same. There are two possible
synchronous products.

\paragraph{Compile time synchronous product.} With this option, the MultitaskC compiler evaluates a static
synchronous product where all communications and interleaving is done a compilation time. So the result is
C code with a maximum performance but complexity may explode.

\paragraph{Runtime synchronous product.} With this second option, the MultitaskC compiler generates statement
tables, data structures and a generic engine to evaluate the synchronous product at runtime. In this case
generated code complexity is linear according to the number of tasks. So execution is slower than with
previous option but complexity cannot explode. Note that also dynamic tasks creation or deletion should
be possible even if not currently implemented.

\section{Comparison with classical features}

The common use of multi-tasks operating systems involves habits for application design and coding. It is
possible to keep main of these habits with MultitaskC.

\paragraph{Scheduling.} It is very common to build an application on top of a fixed priority scheduler
so each task has a priority. A high priority task must be run as fast as possible and keep processor
until it explicitly release it. This is a common implementation as we said before, for instance in VRTX,
LynxOS and VxWorks. This is possible to do fine scheduling with MultitaskC by using rendezvous, a high
priority task can allow other less priority ones to progress only when it is idle. We will seen an
example in the evaluation application for motor control. The number of control rendezvous in low priority
tasks allows fine scheduling and with compiled parallelism this is at no cost.

\paragraph{Communications.} Communication schemes can be implemented with rendezvous and eventually
synchronous broadcast. In~\cite{Delchini:95} we have implemented message queues similar to VRTX ones
and also counting semaphores. It is possible with simple preprocessor macros to implement communications
with a syntax close to the operating systems versions so source code is not very different.

\section{MultitaskC programming remarks}

The "group" instruction can be used to reduce synchronous product result size, by grouping control
statements for example. Errors cases can be grouped to form atomic processing because when an error
occur it is not necessary that actions should be interleaved with other tasks.

The runtime synchronous product can be used during application development to get faster results
if it's possible. Then, when the application is ready, compiled synchronous product can be used for
better performances. Note that the two products have the same semantics but runtime result is
different because in one case everything is compiled and in the other one, some software latency
is introduced to evaluate communications and interleaving.

It's possible to build an application with MultitaskC under UNIX for example. So the developer can simulate
target hardware under UNIX and when the application is ready, compile for the final target.

% ------------------------------------------------------
\chapter{Evaluation}
\label{sec:evaluation}

In this section we are going to point out evaluations made in~\cite{Delchini:95} for data acquisition system
and another evaluation made during 2006 on an 3 axis cobot with motor control and a more recent and more
detailed one on a steer-by-wire application with 1 axis motor control and force feedback.

\section{Previous evaluations}

The first evaluation was made in~\cite{Delchini:95} on data acquisition systems with VRTX and LynxOS compared
to MultitaskC. The main results are that interrupt latency is better with MultitaskC and also that it is possible
to recover ADC conversion dead-time with MultitaskC but not with any operating system. The dead-time recover
is possible because of zero cost context switching in MultitaskC. When an event occur in a part of a detector
the hardware data acquisition system is triggered and mainly ADC conversions starts and then data readout
software can operate. The conversion time can be significant and the naive way to wait for it is resulting
in dead-time. The idea is to recover dead-time by doing something useful during ADC conversion. Of example
in a particle physic data acquisition system, the user can do data filtering on the previous event during
ADC conversion of current event. But as the conversion time may vary depending on the event complexity,
it is essential to be able to start readout very quickly, this is possible with MultitaskC because switching
from a data filtering task to a data readout one is very fast. With an operating system, this is possible
but context switching may introduce more dead-time.

A second evaluation was made with a 3 axis cobot and impedance control, one version with VxWorks on a PC104
IA32 motherboard. An peripheral dedicated hardware board was implementing a current control loop with a
proportional corrector. The main processor was implementing a position and speed control loop that should
be as fast as possible to get realistic feeling on the cobot. The VxWorks version was able to run with
a 2 milliseconds cycle time and the MultitaskC version was running at 250 microseconds cycle time. More than
giving better feeling, the speed gain was interesting for speed elaboration based on position sensor. It
is a common thing in motor control to use a position sensor only without a speed sensor to reduce cost
,weight and congestion. The speed sensor can be interpolated from positions and cycle time. But this has
to be filtered because of typical very low speed during cobot operation. Speed filtering with a median
filter is giving really better results but need some computations that take some time. And the faster
speed elaboration is the better as usual in signal processing. So in this case the MultitaskC version was
8 times faster than the VxWorks one and thus speed elaboration was really better.

\section{Evaluation application description}

The last evaluation was made for an electrical motor control application with force feedback steer-by-wire.
The system is made of a wheel with a position sensor and a 24V motor. An H-bridge power board is driven
by a PWM (Pulse Width Modulation) generator integrated in a Freescale dual core PowerPC processor. The
processor also integrates a ADC for motor current sensor and a quadrature position logic to get position
sensor value. Usually, as in previous application, current control loop is made by dedicated hardware.
In our application, we will use one core for motor control with two tasks, one for position/speed control
and the other one for current control. The other core will be dedicated to communications via an Ethernet
link with a car simulator that will be coupled by position exchange for force feedback.

The two control loop must as usual run as fast as possible and the network activity must run at the same
rate than the position/speed loop because of position information exchange. Typically, the current loop
should run at 10 microseconds cycle time and position/speed one at 100 microseconds again for better
speed evaluation.

\section{Language/OS version}

The operating system for this application is PharOS. We have implemented the two control tasks as agents,
one with a 100 microseconds clock and the other one with the fastest clock. The fastest clock for current
agent is 34 microseconds otherwise there is a deadline overrun and the system stops. Communication
between the two tasks for current set point is made via a simplified temporal variable.

One important thing is that the ADC conversion time for current loop is typical about 25 microseconds. We
have implemented two version of the current loop, one that is waiting for ADC conversion at each cycle
and another one is reading conversion only is it is ready and keep the previous value if it is not. With
PharOS there is no difference between the two version because best cycle time is always longer than
ADC conversion. This involves that no dead-time recovery is possible here.

\section{MultitaskC version}

The MultitaskC version is also made of two tasks, one with a 100 microseconds cycle time and the other one
can run up to 26 microseconds for the version that waits for the ADC to be ready at each cycle. And up
to 9 microseconds for the version that is not waiting for ADC conversion. Communication between the two
tasks is made by a shared memory.

We have used fine synchronous product tunning in this application. The current loop have a higher priority
than the position/speed one. This is implemented by using a scheduling rendezvous, the low priority task
is asking the highest one to grant progression frequently in its execution flow. The highest priority
task is granting the scheduling rendezvous only when it is idle : waiting for ADC conversion or waiting
to reach next deadline. This fine scheduling is tuning the synchronous product expression interleaving
so performance are better than with the default interleaving.

\section{Results comparison}

The MultitaskC version in both case is better. It is possible to recover dead-time and for example do better
filtering for speed evaluation or safety. For example, it is possible with the MultitaskC version to
check redundant sensors for failure recovery and this while motor control is made at ideal rate. It is
also possible to communicate with a redundant processor that will become active in case of failure
of the local one. This is very important for a steer-by-wire application, very good quality force feedback
is mandatory and safety also.

% ------------------------------------------------------
\chapter{Conclusion}
\label{sec:conclusion}

\section{Conclusions about comparisons}

The conclusion points listed here are given by several comparison results. They are based on multiple applications
programmed with different methods. We think that it is possible to generalize several conclusions and consider that
some results are pertinent whatever the context. Note that some points in this conclusion are concerning all
CPLs rather than just MultitaskC.

As in~\cite{Delchini:95} we can remind that the time taken to react to an interrupt is just hardware dependent
and for that there is no benefit with MultitaskC than with operating systems. We just what to point out that
with a CPL, it is possible that interrupts are always open during execution. This is different with an operating system
in which interrupts must be disabled for some times to insure system coherency. As an example, during scheduling
interrupts are disabled to permit this operation atomically.

As we sensed at the beginning of this work, MultitaskC allows to recover dead time consumed by an operating system.
Particularly, in an operating system, the context switching handling is really time consuming and all communications
objects used for data exchange and tasks synchronization are done dynamically involving a system latency that can
be prohibitive for some applications. Tasks activations in response to an interrupt is also better with MultitaskC
than with an operating system.

The fact that, with MultitaskC, the optimizer of the target compiler has a vision of all tasks in one source function
is also a benefit for applications. This can be compared to inter procedural optimizations, but there we can talk of
inter tasks optimizations. This global optimization can be done with any target processor. All these benefits should
be considered with keeping in mind that it is possible to implement an application with MultitaskC using most
traditional programming habits and concepts. So the programmer which is usually familiar with operating systems
services and high level programming languages can use MultitaskC easily with a short period of adaptation.

These response time improvements and the static compilation of parallelism handling brought by MultitaskC allows to
get a better benefit of hardware specific features than with operating systems. We've shows that in~\cite{Delchini:95}
concerning dead time recovery. With operating systems it is possible to implement this kind of principles but the
limitation comes really early compared to MultitaskC due to system software latency.

Note that in~~\cite{Delchini:95}, the MultitaskC version of the application are similar whatever the processor target
is. The operating system versions also also quite similar but less than the CPL ones. MultitaskC brings easier
portability and even the generated code may be reusable for targeting other processors if any hardware dependent
code is encapsulated.

\section{MultitaskC scope}

We have shown that with MultitaskC it possible to keep programming habits that comes from the use of traditional
operating systems and languages. We have also demonstrate that MultitaskC brings actual advantages in terms of
execution efficiency and portability. It was also shown that MultitaskC allows the use of hardware features with
more performances than Operating systems or in other words that the performance limit with MultitaskC is given
by hardware without software latency.

However, it is obvious that such a programming language is not suitable to solve all problems. We are going to
raise the main characteristics of problems for which MultitaskC can offer an interesting solution.

Our language should be particularly suitable to develop embedded applications on targets with few resources and small
memory configurations. The main benefit is to be able to specify the application as communicating tasks when an operating
system is not usable. If time constraints are not important, runtime synchronous product can be used. The only restrictions
are then the size of the application that are limited by hardware only.

\section{Possible language extensions and further work}

MultitaskC programs modeled as communicating processes can allow to verify properties with automated proof systems. This
is the main purpose of synchronous languages we have described before (Esterel, ...). As we said before, for other CPLs, there
are tools that have been developed at INRIA and also at LaBRI for automatic checking of properties on synchronous
product results. At INRIA, these tools are based on an standardized automata format called OC for "Object Code" (introduced
by~\cite{Halbwachs:91}). Esterel and Lustre can for example produce results in OC format and then verification tools
based on OC can be used and considered as common for the two languages.

It should be interesting to produce OC code as a result of an MultitaskC compilation so it will be possible to use
existing tools for automated proof. We think that it is possible to do so even if there are semantic differences between
MultitaskC and other languages of the Esterel family.

Tasks priorities handling is possible currently with MultitaskC. However, the programmer is responsible for slicing low
priorities tasks and processing relinquish in high priorities ones. This can be partially done automatically by the
compiler if we add a priority notion in the language. The possibility to do fine tunning of the synchronous product must
be kept but the programmer should have an easier way to express it. A low priority task could be sliced automatically with
maximum interleaving and then the "group" statement can be used to tune granularity.

The fact that with MultitaskC it is possible to recover dead time and have performances that are only limited by hardware,
could be advantageously used for safety. For example, in the motor control application we described before, one can use
ADC conversion dead time to communicate with a redundant processor for safety purpose. This is a way we would like to
investigate in future works. It can also be combined with the possibility to use the execution model generated
by synchronous product for safety and critical aspects of an application.

% ------------------------------------------------------

% ------------------------------------------------------
% References
% ------------------------------------------------------
\scriptsize
\bibliographystyle{latex8}
\bibliography{report}

\end{document}
